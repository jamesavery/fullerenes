\documentclass[article,preprintnumbers,amsmath,amssymb]{revtex4-1}
\usepackage{graphicx}
\usepackage{dcolumn}
\usepackage{color}
\usepackage{mathrsfs}
\newcommand{\C}[1]{\ensuremath{\mathrm{C}_{#1}}}

\begin{document}

\title{PROGRAM FULLERENE (Version 4.0)\\ -- A Program for the Topological Analysis of Fullerenes --\\ USER'S MANUAL}

 \begin{figure}[htbp]
   	\begin{center}
  		\includegraphics[width=0.6\textwidth]{Toc.png} \\
                 \label{pic:toc}
   	\end{center}
 \end{figure}

\author{Peter Schwerdtfeger}
\email[]{p.a.schwerdtfeger@massey.ac.nz}
\author{Lukas Wirz}\email[]{l.wirz@massey.ac.nz}

\affiliation{Center of Theoretical Chemistry and Physics, The New Zealand Institute
for Advanced Study, Massey University Auckland, Private Bag 102904,
North Shore City, 0745 Auckland, New Zealand.}

\author{James Avery}\email[]{j.e.avery@massey.ac.nz}

\affiliation{Copenhagen, Denmark.}
\date{\today}

\maketitle

\clearpage


{\bf Important Copyright Message:}\\\\
{\it This is an} {\bf open-source code} {\it and you may use and modify the program at your own will.
If you like to know why, read for example} Darrel C. Ince1, Leslie Hatton, 
John Graham-Cumming, Nature {\bf 482}, p.485 (2012). {\it We kindly ask you that if you 
use our program and subsequently publish data to cite the three references given below.
Note: before you distribute the program to other agencies or users outside your
department we kindly ask you pass the information on that new users should be
added to our user database. For details see our website at:}\\
http://ctcp.massey.ac.nz/index.php?group=\&page=fullerenes\&menu=fullerenes\\\\

{\bf Please cite the following papers if you used this program for publishing data:}\\\\
1) P. Schwerdtfeger, L. Wirz, J. Avery, {\it Topological Analysis of Fullerenes - 
A Fortran and C++ Program (Version 4.0)}, (Massey University Albany, 
Auckland, New Zealand, 2012).\\
2) P. W. Fowler and D. E. Manolopoulus, {\it An Atlas of Fullerenes}
(Dover Publ., New York, 2006). This book is highly recommended. 
It helps understanding how this program functions.  Many of the concepts used 
can be found in this book. \\
3) D. Babic, {\it Nomenclature and Coding of Fullerenes}, J. Chem. Inf. Comput. Sci. {\bf 35}, 515-526 (1995).\\\\

{\bf Further reading:}\\\\
4) Z. C. Wu, D. A. Jelski, T. F. George, {\it Vibrational Motions of
Buckminsterfullerene}, Chem. Phys. Lett. {\bf 137}, 291-295 (1987).\\
5) D. E. Manolopoulus and P. W. Fowler, {\it Molecular graphs, point groups, 
and fullerenes}, J. Chem. Phys. {\bf 96}, 7603-7614 (1992).\\
6) G. B. Adams, M. O'Keefe, and R. S. Ruoff, {\it Van der Waals Surface Areas
and Volumes of Fullerenes}, J. Phys. Chem. {\bf 98}, 9465-9469 (1994).\\
7) W. O. J. Boo, {\it An Introduction to Fullerene Structures},
J. Chem. Ed. {\bf 69}, 605-609 (1992).\\
8) D. Babic, D. J. Klein and C. H. Sah, {\it Symmetry of fullerenes},
Chem. Phys. Lett. {\bf 211}, 235-241 (1993).\\
9) T. Pisanski, B. Plestenjak, A. Graovac, {\it NiceGraph Program and its 
applications in chemistry}, Croatica Chemica Acta {\bf 68}, 283-292 (1995).\\
10) B. Plestenjak, {\it An algorithm for drawing Schlegel diagrams}, http://www-lp.fmf.uni-lj.si/plestenjak/Papers/NICEGR.pdf.\\
11) J. Bondy, U. Murty, {\it Graph Theory} (Springer, Berlin, 2008).
12) A. J. M. Wilson, {\it Graphs and Applications. An Introductory Approach} (Springer, Berlin, 2000).\\
13) J. Cioslowski, N. Rao, D. Moncrieff, {\it Standard Enthalpies of Formation of Fullerenes and Their
Dependence on Structural Motifs}, J. Am. Chem. Soc. {\bf 122}, 8265�8270 (2000).\\\\
     
{\bf Acknowledgement}\\\\
PS is indebted to the Alexander von Humboldt Foundation (Bonn) for financial support 
in terms of a Humboldt Research Award, and to both Prof. Gernot Frenking and 
Dr. Ralf Tonner (Marburg) for support during his extended stay in Marburg where 
writing of this program began. We acknowledge the help of Darko Babich, Patrick 
W. Fowler and David E. Manolopoulus to give us the permission to freely distribute 
their Fortran subroutines.\\\\
 
 \begin{figure}[htbp]
   	\begin{center}
  		\includegraphics[width=1.0\textwidth]{C840.png} \\
                 \label{pic:toc}
   	\end{center}
 \end{figure}

\clearpage

\section{Introduction}
The Fullerene Program creates cartesian coordinates for different fullerenes isomers and performs a topological/graph 
theoretical analysis of the 3D structure. The results can be used for plotting 2D fullerene graphs (Schlegel diagrams) 
and 3D structures, and as a good starting point for further quantum theoretical treatment. The program was originally 
written in standard FORTRAN and is can easily be implemented in LINUX/UNIX environment. It provides also input files 
to external plotting programs. In the current version only regular fullerene (i.e. consisting of pentagons and hexagons) 
fulfilling Euler's theorem are considered. The program constructs the 3D structure in cartesian coordinates from the 
the canonical ring spiral pentagon indices through either a Tutte embedding (TE) or the adjacency matrix
eigenvector (AME) method. It can also construct the n-th leapfrog and halma fullerenes from
Goldberg-Coxeter transformations of \C{20} or \C{60}. From a general isomer structure one can perform a
Goldberg-Coxeter transformations or one can apply vertex insertions and Stone-Wales transformations. 
The Wu force-field and geometry optimization using a Fletcher-Reeves-Polak-Ribiere
minimization with analytical gradients is implemented in the current version, providing 
a good initial guess for cartesian coordinates. Schlegel diagrams (2D graphs)
can be produced using a variety of different algorithms. The program calculates the volume and surface area 
of a fullerene (irregular or not) through tesselation in trigonal pyramids and calculates measures for
spherical distortion and convexity. It further calculates the minimum covering sphere, the minimum distance sphere
and the maximum inner sphere. Note that in the case of nonplanar 5- or 6-rings there is no unique definition for the 
volume of a fullerene except for the convex hull. There is no reason however why any other definition than the 
fast tesselation algorithm used here should be preferred. The program further calculates the number of Hamiltonian cycles
and produces ring spiral indices. Finally, Cioslowski's scheme for the calculation of the heat of formation for IPR
(independent pentagon rule) fullerenes is implemented \cite{Cioslowski2000}, as well as Babi\'c's scheme for calculating the total resonance energy
of a general fullerene.\cite{Babic1997}\\

This program works for any (distorted or not) regular fullerene (i.e. a fullerene of genus 0 consisting of 
pentagons and hexagons only). The spiral algorithm of Fowler and Manolopoulus used here is not restricted to starting 
from a pentagon or to canonical indices. For a general list of fullerenes see "The House of Graphs" at 
https://hog.grinvin.org/Fullerenes. The program produces a file with default name {\bf cylview.xyz} to be used for plotting 3D 
structures for programs like CYLview \cite{CYLview}, Avogadro, Jmol, or Pymol. Note for using these programs it is important to force-field 
optimize them, otherwise bonds cannot be idendefied if structures are used directly from
AME, LME or 3D-TE algorithms (see below). We recommend CYLview, it is more robust and works for the largest 
fullerenes up to 1000 atoms. The program produces fullerene 2D graphs in latex format, but you can
use any other program which is capable of producing Schlegel diagrams, for example
QMGA. For this an input file is written called {\bf qmga.dat} \cite{Gabriel2008}.\\

Version 4.0 (released July 2012) now incorporates C++ routines embedded into the original FORTRAN program using 
much improved algorithms compared to the older version. The reason for using two program languages is
that PS is good in old-fashioned Fortran and JA is good in C++. Also some of the original routines used were available 
in Fortran only. Some standard routines from Mathematical Recipes were modified here for the purpose of matrix diagonalization 
and geometry optimization. This program is under permanent construction. The program has been tested for bugs.
Nevertheless, if you have any problem or find a bug please report to one of us.\\

Not implemented yet and on the to do list (in progress) is:

1) A subroutine to fit the minimum outer ellipsoidal cover useful for  
rugby ball like fullerenes and close packing of ellipsoids (coming soon);

2) Use the general Goldberg-Coxeter construction for fullerenes (coming soon);

3) Geometry optimization using the extended Wu-Fowler force field (coming soon);

4) Frequency calculations from the force field optimized geometry;

5) Construction of non-ring-spiral isomers using the generalized ring spiral algorithm (coming soon);

6) Symmetry labels for H\"uckel orbital energies;

7) Extend to non-regular fullerenes of genus 0 (heptagons and squares);

8) Extend to non-regular fullerenes of genus 1;

9) Symmetrize coordinates to point group symmetry;

10) Restart option for subroutine for Hamiltonian cycle;

11) Use of "House of Graphs" fullerene database (coming soon);

12) Plotting program (other than latex) for producing Schlegel diagrams and corresponding duals (coming soon).

\clearpage

\section{Installation and running the program}
All fortran files are in the directory "source" and all C-files are in the directory "libgraph".
The program is LINUX/UNIX based and works fine with {\bf gfortran} and {\bf gc} compilers. You need to use the Makefile included in 
the fullerene.zip file provided and type\\\\ 
{\bf make}\\\\
It currently compiles in a 64 bit version, but you can change to 32 bits in the Makefile if necessary.
The executable "fullerene" runs on a LINUX/UNIX as\\\\
{\bf ./fullerene $<$inp $>$out}\\\\
A number of test input files can be found in the directory "input". If you type\\\\
{\bf make tests}\\\\
it runs all the input jobs and puts them into *.out in the output directory. If you type\\\\
{\bf make clean}\\\\
all the object files are deleted. More is deleted if you type\\\\
{\bf make distclean}\\\\
If you use the fullerene database, the file needs to be in the directory
where the source and libgraph directories are situated.


\section{Program Structure}
The main program ({\bf main.f}) calls a number of subroutines for certain tasks.
Subroutine {\bf DataIn} manages the input and determines the main tasks to be done.
The input is in {\it Namelist format}. 
Important steps in the program are given in the flow diagram below and are
explained in detain in the following.\\

 \begin{figure}[htbp]
   	\begin{center}
%  		\includegraphics[width=1.0\textwidth]{flowdiagram.png} \\
                 \label{pic:flowdiagram}
        \caption{Flow diagram for main program tasks}
   	\end{center}
 \end{figure}
 
\subsection{Create a 3D structure for a specific fullerene}
The easiest way to create a 3D structure is to read in cartesian coordinates (Subroutine {\bf CoordBuild}) for a specific fullerene 
(see files {\it c20.inp} to {\it c540.inp}), or to construct them internally for the high symmetry 
$Ih-$\C{20} or $Ih-$\C{60} isomers (files {\it ico.inp}, {\it icoExp.inp}, {\it icoideal.inp}), or to 
get the cartesian coordinates from input ring spiral pentagon indices by using either the Fowler-Manopoulus 
AME algorithm (the adjacency matrix eigenvector algorithm) \cite{Atlas}, or the more reliable Tutte embedding algorithm (3D-TEA) 
(used by files {\it pentagon1.inp} to {\it pentagon25.inp}). The program sets barycenter of the fullerene 
to the origin. The book ``Atlas of Fullerenes" Fowler and Manopoulus \cite{Atlas} is recommended for the 
use of ring spiral pentagon indices and and H\"uckel (adjacency matrix) $P$-type eigenvectors to construct the 3D fullerene
graph. Note it is critical to get the right H\"uckel eigenvectors for the construction of cartesian coordinates. 
The 3 $P$-type vectors may need to be read in (see file {\it pentagon8.inp} for such an example).
If the AME algorithm fails you can use our Tutte embedding algorithm (which to our opinion is less troublesome 
and may be used as the standard method to construct 3D structures). It is important that the end-product is 
viewed by a molecular visualization program. We recommend CYLview by Claude Legault \cite{CYLview},
Avogadro \cite{Avogadro}, JMol \cite{JMol} or Pymol \cite{Pymol}. These codes are all freely downloadable.
For this purpose a file is written out to cylview.xyz to be used as an input file for these programs. 
Using these coordinates the program calculates the smallest and largest cage diameters which gives already a measure 
for distortion from spherical symmetry (Subroutine {\bf DIAMETER}). It produces the distance matrix if print level 
is set to high (Subroutine {\bf DISTMATRIX}).

 - Use program SPIRAL of Fowler and Manopoulus for creating a ring spiral.
    It also produces canonical ring spiral pentagon indices 
    (Subroutine SPIRALSEARCH) if cartesian coordinate input is chosen.

-  Print all isomers and perform analysis introduced mostly in the book 
    written by Fowler and Manopoulus (ref.2), e.g. pentagon indices and 
    pentagon number, hexagon indices and strain parameter, NMR information
    and number of distinct Hamiltonian cycles if required.

 - Perform Hueckel analysis (Subroutine HUECKEL). 
    This gives you a good hint if the fullerene is open or closed shell.
    Note: It is known that for fullerenes the Hueckel analysis is not
      very reliable as hybridization with the C(2s) orbitals occur due
      to nonplanarity. Hence the sigma-pi separation breaks down.
      Nevertheless, we adopt  alpha=-0.21 au  and  beta=-0.111 au obtained
      from the exp. ionization potential (7.58 eV) and excitation energy
      (3.02 eV) of C60 (note the electron goes into the second LUMO of 
      t1g symmetry). This gives also orbital energies for C60 in reasonable 
      agreement with DFT Kohn-Sham orbital energies. The fullerene might
      also Jahn-Teller distort adopting a singlet ground state instead
      of one of a higher multiplicity. Such electronic effects are not
      captured in this program, and one needs to perform a proper quantum
      theoretical calculation.

 - Use program HAMILTON of Babic (ref.3) for Hamiltonian cycles and IUPAC 
    Nomenclature. The number of Hamiltonian cycles given has been checked 
    against a second algorithm, so it should work. Note, the number gives 
    all distinct Hamiltonian cycles and left-right cycles counted as the 
    same. Although finding all Hamiltonian cycles by the back-track algorithm
    of Babich used here is a NP-complete problem, it works fine up to 
    about C100. After that it becomes computationally very demanding. 
    Note also that the existence of Hamiltonian cycles for fullerenes is 
    only conjectured, and only for layered fullerenes (e.g. fullerene 
    nanotubes) it has been proven to exist, our calculations show that
    they exist for all fullerene isomers up to C100.

 - Establish connectivities between atoms either from given cartesian
    coordinates or from adjacency matrix if ring spiral as input is used:
    1) Bond between atoms, 2) Vertices (Subroutine CONNECT).
    Identify all closed 5- and 6-ring systems (Subroutine RING).
    This routine also determines if Euler's theorem is fulfilled.

 - Determine the center for each 5- and 6-ring (Subroutine RINGC)
    This is required for the trigonal pyramidal tessellation to obtain
    the volume and surface.  This routine also analyzes all 2- and 3-ring fusions
    It further gives the Rhagavachari-Fowler-Manoupoulos neighboring pentagon 
    and hexagon indices as described in the Fowler and Manolopoulos book
    (ref.2). From the hexagon indices one derives if the fullerene fulfills the
    IPR or not.

 - Fletcher-Reeves-Polak-Ribiere geometry optimization using analytical 
    gradients for the Wu force field (Subroutine OPTFF).
    It is very fast, even for C840. Note that the force field optimization
    might distort the fullerene from the ideal point group symmetry.
    On the other hand, the construction of the fullerene by using
    pentagon indices leads to a more spherical arrangement in both
    algorithms (Fowler-Manoupoulos or Tutte), e.g. barrels instead of
    nanotubes.

 - Calculate the volume of the fullerene by summing over all
    tetrahedrons spanned by the three vectors (Subroutine VOLUME).
     CM-CR  (center of cage to the center of ring)
     CM-CA1 (center of cage to atom 1 in ring)
     CM-CA2 (center of cage to atom 2 in ring)
    There are 5 such tetrahedrons in a 5-ring and 6 in a 6-ring
    Note that CM is already in the origin
    Let CR=(X1,Y1,Z1) , CA1=(X2,Y2,Z2) , and CA2=(X2,Y2,Z2)
    Then the volume V for a irregular tetrahedron is given by 
    the determinant

                                 | X1 Y1 Z1 |
     V = abs(Vdet)  ,   V =  1/6 | X2 Y2 Z2 |
                                 | X3 Y3 Z3 |

   Calculate the surface area A and the area/volume ratio (Subroutine VOLUME)

                                        2              2              2
                             | Y1 Z1 1 |    | Z1 X1 1 |    | X1 Y1 1 |   
     A = 1/2 d**0.5 ,   d =  | Y2 Z2 1 | +  | Z2 X2 1 | +  | X2 Y2 1 |
                             | Y3 Z3 1 |    | Z3 X3 1 |    | X3 Y3 1 |

   Note that the ideal C60 coordinates can be constructed from scratch 
    (Subroutine COORDC60). This routine was constructed to test the program 
    for the case of an ideal capped icosahedron, where the analytical formula 
    is well known, i.e.   V=[(125+43*sqrt(5))*R**3]/4
    and R is the distance between the atoms (all the same)
    Setting R=Rmin (Rmin is the smallest distance in C60) this gives a
    lower bound for the volume, while the volume of the covering central sphere 
    gives the upper bound.
    For two different bond distances, R5 for the 5-ring and R6 for the 6-ring
    joining another 6-ring, the volume can be determined as well after some tedious
    algebraic manipulations:
    i.e.   V=5[(3+sqrt(5))*(2R5+R6)**3]/12-[(5+sqrt(5))*R5**3]/2
    For C20 (ideal dodecahedron) we have V=[(15+7sqrt(5))*R5**3]/4
    For C20 and C60 the result of these formulae are also printed.
   This method gives sensible results for convex fullerenes.

 - Calculate the minimum covering sphere (MCS) of the cage molecule 
    (Subroutine MINCOVSPHERE): The MCS in m-dimensional space exists, is unique 
     and can be expressed as a convex combination of at most (m+1) points, hence
     our algorithm stops when 4 points are left over in the iteration process.

    The problem can be reduced to

    min(c) max(i) || p(I) - Cmcs ||

    where ||..|| is the Euclidian norm in m dimensions and Cmcs is the center 
    of the MCS.

    Note: The spherical central cover SCC is not the minimum covering sphere MCS
    (except if all distances from the center of points CM are the same as in the
    ideal capped icosahedron). The spherical central cover is taken from the
    CM point with radius Rmax (longest distance to one vertex).
    The minimum covering sphere is calculated at the end of the
    program using the algorithm of E. A. Yildirim, SIAM Journal on Optimization
    Vol. 19(3),1368-1391 (2008) and the test by T. H. Hopp and C. P. Reeve,
    NIST, US Department of Commerce (1996)'). Note that the much simpler algorithm 
    by F. Lu and W. He, Global Congress on Intelligent Systems (2009),
    DOI 10:1109/GCIS:2009:381, simply does not work for more than m+1 points on a 
    surface or close by as their linear equation becomes linearly dependent.
    Note also that the function Psi in Yildirim's algorithm is really the function
    Phi defined earlier in his paper in section 2 and corrected in the SIAM paper.
    His easier to program algorithm 1 was also tested, but is much slower.
    If the value given in the iteration as "convergence" is close to
    zero (equal zero), the iteration stops (if it falls below epsilon).
    You can change the epsilon parameter in subroutine Sphere.
    You can also try algorithm 1 of Yildirim through subroutine Sphere1
    which is included in an file called algorithm1.f (although this file has not
    been updated and further developed). Note that we changed the first
    condition in this algorithm by choosing the furthest point from CM.
    In the final statistics there should be 0 points outside the sphere
    and at least 1 point on the sphere.
    At the end the Van der Waals radius of carbon (1.415 Angstroems) is added to the
    radius of the minimum covering sphere (note input coordinates for this need to be
    in Angstroems otherwise change the program), and the volume of
    an ideal fcc solid is calculated. The Van der Waals radius is chosen such that
    for C60 the solid-state results of P.A.Heiney et al., Phys. Rev. Lett. 66, 2911 (1991)
    are reproduced. The definition of the distortion parameter D from the MCS or for the
    the isoperimetric quotient IPQ is

    $IPQ=36Pi(V^2/A^3)$

    D=[100/(N*Rmin)]* sum(i=1,N) {Rmcs - ||pi-Cmcs|| }    (N=MAtom)

 - Calculate the minimum distance sphere (MDS) of the fullerene.
    The MCS definition for the distortion is biased for the case that few atoms stick 
    out on a sphere and the MDS measure may be more appropriate for a measure
    from spherical distortion. The MDS is defined as
    The problem can be reduced to

    min(Cmds) 1/N sum(i=1,N) | Rmds - || p(I) - Cmds || |

    where ||..|| is the Eucledian norm in m dimensions. Cmds has to lie within the
    convex hull. The MDS may not be uniquely defined, as there can be many 
    (even degenerate) local minima, but for most spherical fullerenes it should 
    just be fine. Analogous to the MCS there will be a measure for distortion 
    from spherical symmetry.

    D=[100/(N*Rmin)]* sum(i=1,N) | Rmds - || p(I) - Cmds || |

 - Calculate the maximum inner sphere (MCS) of the cage molecule

    max(Cmds) min(i) || p(I) - Cmds ||

    The maximum inner sphere is important for evaluating how much space
    there is in a fullerene for encapsulating atoms and molecules. For
    this the radius and volume is printed out with the Van der Waals
    radius of carbon taken off Rmds. 

 - Produce the (X,Y) coordinates of a fullerene graph (Subroutine SCHLEGEL).
    Schlegel projection (SP):
    Here the points are rotated (if in input I1,I2, and I3 are given) so to
    put the selected vertex, edge or ring center on top of the z-axis as the
    point of projection (otherwise the point (0,0,zmax) is chosen with zmax
    being the point with maximum z-value from the original input coordinates).
    Points are then sorted in descending order according to their z-values.
    The circumference for atoms and rings down the z-axis are determined.
    The Schlegel projection is created giving as output the projected (X,Y)
    coordinates. The connections between the points are already written out
    earlier in the output such that the fullerene graph can be drawn.
    There are two choices for the projection, depending if you choose the
    outer ring or the center part of the fullerene graph as a starting point:
    1) The cone projection (CSP), i.e. points are projected out to an enveloping 
       cone and then down to a plane below the fullerene. The input I1,I2,I3 
       defines the center of the fullerene graph. The last ring center should 
       be at the bottom of the fullerene and if detected, will not be projected 
       out, or if not will have a large scale factor (this center may be ignored 
       in the drawing). Also, the last points on the outer ring in the fullerene
       graph are scaled in distance by 1.2 in order to make the outer rings 
       more visible. This also moves the outer centers within the ring.
    2) The perspective projection (PSP), i.e. points are projected down a plane 
       from a set projection point. In this case the input I1,I2,I3 defines the
       peripheral ring of the Schlegel diagram. 
    From the output you can easily construct the name of the fullerene. 
    At the end a rough printout of the fullerene graph is
    produced. Note that this is o.k for fullerenes up to about C100, beyond it
    it becomes too crowded and a proper plotting program should be used.
    Nevertheless, it serves for a first rough picture. 
    Furthermore, for large fullerenes it becomes critical to correctly set the
    projection point or point of the cone. If for example the projection
    point is too far away from the fullerene, edges may cross.
    Other algorithms for producing Schlegel diagrams:
     - Tutte graph with linear scaling (2D-TGE-LS)
     - Spring embedding with barycentric Coulomb repulsion (SE+C)
     - Pisanski-Plestenjak-Graovac embedding (PPGA)
     - Kamada-Kawai embedding (2D-KKE, this gives a 2D picture of a 3D structure
        and has edge crossings).
    
 NB: The algorithm for locating all 5-and 6-rings might not be the smartest
      one, but as this program requires only a second or so to run it was
      not important to find a better algorithm.
 You should also be aware of program fullgen for generating nonisomorphic fullerenes.
      It is written by Gunnar Brinkmann (Gunnar.Brinkmann@Ugent.be) and can be
      downloaded from Brendan McKay's website (Australian National University) at
      http://cs.anu.edu.au/~bdm/plantri/



 For any questions concerning this program please contact P. Schwerdtfeger
   Centre for Theoretical Chemistry and Physics (CTCP)
   The New Zealand Institute for Advanced Study (NZIAS) 
   Massey University Auckland, Bldg.44, Private Bag 102904
   North Shore City, 0745 Auckland, New Zealand
   email: peter.schwerdtfeger@gmail.com
   http://ctcp.massey.ac.nz/   and  http://www.nzias.ac.nz/
   --> Always open to improvements and suggestions

\section{Input description}

 Input and output files are in the folders  input  and   output  respectively.
 This program has been tested for the ideal capped icosahedron (input file ico.inp)
   and for many other fullerenes which are found in the following input files:
       C20 (c20.inp), C24 (c24.inp), C26 (c26.inp), C28 (c28.inp), C30 (c30.inp),
       C36 (c36.inp), C50 (c50.inp), C60 (c60.inp), C70 (c70.inp), C72 (c72.inp),
       C74 (c74.inp), C78 (c78.inp), C80 (c80.inp), C92 (c92.inp), C100 (c100.inp),
       C180 (c180.inp), C320 (c320.inp), and C540 (c540.inp), pentagon1.inp, ...,
       pentagon25.inp
   The coordinates are mostly B3LYP aug-cc-pVDZ optimized up to C60, and
    cc-pVDZ up to C180, and 6-31G for the rest and all for singlet states
    (except of course for the ones where the pentagon indices input is
    chosen). Note that for some of the fullerene coordinates the singlet
    state chosen may not be the electronic ground state.
       -> Many definitions depend on the use of Angstroems, so please use this unit.

 Input (Either Namelist or in Free Format):   (use Angstroem for distances)

 1) Text card of 80 characters (A80 format)
    (you can put as many text cards in as you want, the main title (first card)
     is printed out extra

 2) Input to create cartesian coordinates and main flags for the program
    \&Coord options /       (e.g. \&Coord IC=20, IOPT=1, R6=1.42 /)
    list of options: NA,IC,IP,IV1,IV2,IV3,ixyz,ichk,leap,isonum,IPRC,TolR,R5,R6,xyzname
    NA= Number of Atoms (Default: 60)
    IC= Flag for construction of cartesian coordinates (Default: 0)
    IP= Print option (Default: 0)
    IV1= Number for Hueckel P-type eigenvector for AME algorithm (Default: 2)
    IV2= Number for Hueckel P-type eigenvector for AME algorithm (Default: 3)
    IV3= Number for Hueckel P-type eigenvector for AME algorithm (Default: 4)
    ixyz= Flag for producing input file for CYLview, Avogadro or others in standard
      xyz format (Default: 0)
    isonum= Isomer number according to the scheme of Fowler and Manopoulus (Default: 0)
      If IC=2, 3 or 4 and isonum not zero, than pentagon indices are taken from the
      isomer list contained in a database (see below). There are two databases, one
      for the general isomers (IPRC=2) and one for the IPR isomers (IPRC=1), the
      definition is similar to the IPR parameter below (Default: 2).
    xyzname (max 20 characters) file name if ixyz.ne.0 (default: cylview.xyz)
    TolR= Tolerance in \% (Default: 33)
    R5= pentagon bond distance (Default: 1.455)
    R6= hexagon  bond distance (Default: 1.391)
    In detail:
      If IC = 0 No coordinate input required, cordinates are constructed 
              for the IPR isomer of C60
           In this case only one card is read in:
              R5,R6        (arbitrary units, e.g. Angstroms)
              R5: Bond lengths in the pentagons 
              R6: Bond length of the bonds connecting hexagons
              If R5=R6 chosen then the ideal capped icosahedron is obtained
      If IC = 1 Cartesian Coordinates expected as input
         In this case N lines with    Z, X, Y, Z  in free format are expected.
         (Z= Nuclear Charge, X,Y,C Cartesian Coordinates for Atom).
         NB: Z is not really needed, but you can copy Gaussian output
          directly into the input file
      If IC = 2 or 3 Cartesian Coordinates are created from pentagon 
             ring spiral list. Extra input required (free format):

           IRSP(I),I=1,12 

           R6 is taken as the smallest bond distance in the fullerene and IP(I)
           identify the locations of the pentagons as described in detail
           in ref.2, i.e. IP is the pentagon ring spiral numbering scheme. 
           Note this only works for ring spiral fullerenes as described in 
           detail by P. W. Fowler and D. E. Manopoulus. Use the canonical
           pentagon ring indices if possible (transformation to the canonical
           from should work as well).
           IC=2: AME algorithm using P-type eigenvectors produced from the 
            adjacency matrix:
            If problem with eigenvectors are found to construct the
            cartesian coordinates, i.e. the identification of P-type
            eigenvectors, three integer values IV1, IV2, IV3 can be specified
            identifying the eigenvectors to be chosen. pentagon8.inp is such an example.
            In this case a severe warning occurs which means you should carefully
            check the eigenvectors used and cartesian coordinates produced.
            Otherwise coordinates are obtained which are useless. This is more
            often the case as you might expect. 
          IC=3: Same as IC=2 but the Laplacian matrix is used instead of the
            adjacency matrix (LME algorithm). 
          IC=4: Tutte embedding (3D-TE) algorithm. This should always work,
            although the initial fullerene might be too spherical. But this
            algorithm is easier, and (in theory) should never fail.
           Examples are given in the input files starting with 'pentagon'.
           Please use Angstroems.
      If IP>0 larger output produced, i.e. the full distance matrix, all
          Hamiltonian cycles and all 3-ring connections.
      if leap=n than the n-th leapfrog fullerene is generated.
      Connectivities are found for atoms with distances between
         R6   and   R6*(1+TolR/100)   if cartesian coordinate input is chosen.
      If TolR=0. default value of 33\% is used. 
         NB: If this parameter is set at a value too large, unwanted connectivities
          are produced resulting in smaller polygons. This parameter
          should reflect the maximum deviation in distance from the
          smallest distance found.

 3) Option for force-field optimization:
      \&Opt options /        (e.g. \&Opt Iopt=1 /)
      list of options: Iopt,ftol,WuR5,WuR6,WuA5,WuA6,WufR,WufA,fCoulomb
      Iopt= Flag for force-field optimization (Default: 0)
      In detail:
       If Iopt=1  then fullerene is optimized using the force field method
         of Wu et al within a Fletcher-Reeves-Polak-Ribiere algorithm:
         Z.C.Wu, D.A.Jelski, T.F.George, Chem. Phys. Lett. 137, 291-295 (1987).
         Note that there are more sophisticated force fields available,
         but for these general parameters for fullerenes are not
         yet available, and the Wu force field does the job to create good initial
         cartesian coordinates for further refinement using more sophisticated
         QM methods. Note that a converged energy much greater than zero implies
         that the set distances and angles in the Wu force field cannot be
         reached for all atoms and rings. For further reading see: 
         A.Ceulemans, B.C.Titeca, L.F.Chibotaru, I.Vos, P.W.Fowler, 
         J. Phys. Chem. A 105, 8284-8295 (2001).
       If Iopt=2  Preoptimize with input force field, then optimize with Wu
         force field. This is especially usefull for fcoulomb input (see below).  
         NB: Avogadro has a more sophisticated force-field which you can try out.
       ftol: The convergence tolerance on the function value is input as ftol
         (Default: 5.0E-8)
       WuR5,WuR6,WuA5,WuA6,WufR,WufA: Force field parameters for Wu force field for
         distances R5, R6, angles A5, A6, force constants for distance WufR and
         angles WufA (see paper by Wu et al. for details).
         Defaults: WuR5=1.455, WuR6=1.391, WuA5=1.08d2, WuA6=1.2d2, WufR=1.d6,
                   WufA=1.d5
       If fCoulomb>0. then add an additional repulsive Coulomb forct from the 
         barycenter to the atoms      (Default:  fCoulomb=0.d0)
         This is extremely useful for an initial geometry optimization to keep
         the cage convex if for example the Tutte construction leads to not
         a good guess of the initial structure. In such cases fCoulomb=1.d2
         is a good choice.

 4) Option for calculating Hamiltonian cycles and IUPAC numbers 
      \&Hamilton options /        (e.g. \&Hamilton IHam=1 IUPAC=1 /)
      list of options: IHam,IUPAC
      In detail:
      If IHam>0 Then Subroutine HAMILTON is called.     (Default: 0)
         IHam=1 Routine will stop after 1 million Hamiltonian cycles are found
         IHam=1000000000 Program runs forever and prints if IHam is reached
      If IUPAC=1 IUPAC numbers are produced.   (Default: 0) 
         Note only with this option together with IP=1 in \&Coord input 
         all Hamiltonian cycles are printed out. IP=0 only gives the
         best cycle (see Babic, ref.3).
         IUPAC=0 goes into a fast subroutine and prints only the total number of
         Hamiltonian cycles.

 5) Option for producing list of isomers and properties.
      \&Isomers options /        (e.g.\&Isomers IPR=1, IPH=1 /)
      list of options: IPR,IPH,IStop,IChk,chkname (Default 0 for all options
                                                   and 'checkpoint' for chkname)
      In detail:
      If IPR>0 then the ring spiral subroutine of Fowler and Manolopoulus is used.
         This sub-program catalogues fullerenes with a given number of
         vertices using the spiral algorithm and a uniqueness test
         based on equivalent spirals. The required input is IPR.
         IPR=1 for isolated-pentagon isomers from C60 onwards.
         IPR=2 for general isomers (note that this generates a lot of output and
            takes some computer time for fullerenes larger than C60).
      If IPH=1 then number of distinct Hamiltonian cycles is calculated for
          every isomer (note this is computer time extensive).
      If istop=1 program stops after calling this routine.
      If IChk=1  Restart: Isomer list is continued from previous output file 
          called 'checkpoint' as default if not otherwise give in chkname.
          This is a restart option from a previous run which terminated. Note
          that the new output file does not contain the previous one.
      The resulting output is a catalogue of the isomers found containing
         their idealized point groups, canonical spirals, and NMR patterns
         (see ref.2).

 6) Option for producing coordinates for fullerene graphs (Schlegel diagrams).
      \&Graph options /      (e.g. \&Graph IG=1, ISO1=1, ISO2=3, ISO3=7 /)
      list of options: ISchlegel,ISO1,ISO2,ISO3,PS,SCALE,SCALEPPG
      We recommend IG= 1, 2, 4 or 7
      In detail:
      If ISchlegel>0 Use Program Schlegel for generating fullerene graphs
         ISchlegel=1 Use the perspective projection method (PSP).
         ISchlegel=2 Use the cone projection method (CSP).
         ISchlegel=3 Produce the Tutte graph (2D-TE)
         ISchlegel=4 Produce the Tutte graph and perform linear scaling (2D-TE-LS)
              (scale factor can be read in).
         ISchlegel=5 Produce the Tutte graph and perform spring embedding optimization
              (rij-r0)**2 with r0 set to 2.0 (2D-SE).
         ISchlegel=6 Starting from the Tutte graph perform spring + repulsive Coulomb
              force embedding optimization (2D-SE+C). The repulsive force is taken
              from the barycenter to the vertex. r0 is set to 2.0. The force
              constants are set such that the graph looks nice.
         ISchlegel=7 Starting from the Tutte graph perform a Pisanski-Plestenjak-Graovac 
              embedding PPGE) optimization (called Schlegel by the authors)
              Note: This is much faster than their simulated annealing algorithm.
         ISchlegel=8 Starting from the Tutte graph perform a Kamada-Kawai embedding 
              optimization using the distance matrix (2D-KKE).
              Note: This gives a 2D picture of a 3D structure, thus
               it is not related to a Schlegel diagram and has edge crossings.
      If ISO1=0 Use the input coordinates for the construction of 
               the Schlegel diagram.
      If ISO1.ne.0 Specifying the ring center, edge or
              vertex through which the z-axis goes at the top of
              the projection (under construction). This rotates
              the fullerene around the origin of points into the
              right position for the Schlegel diagram. Since 3 atoms
              uniquely define the ring, two the edge, and one the vertex,
              the input is three integers with the corresponding
              atom numbers  IO1, IO2, and IO3, i.e. for these values
                    1 0 0 is a vertex with atom 1 on the top of the z-axis;
                    7 8 0 is an edge between atom 7 and 8 and z-axis goes
                          the middle of the bond;
                    12 13 50 is a ring (either pentagon or hexagon determined
                          by the program) and z-axis goes through the center
                          of the ring;
              NB: You can run the program first with 0 0 0, check the positions
                  and run it again with the appropriate values.
              NB2: For IG=1 the input (if chosen) requires to be a ring, i.e.
                  IO1, IO2 and IO3 are required, otherwise they are chosen by the
                  program using the largest z-coordinate.
      If Scale.ne.0. Scaling factor for linear scaling of Tutte graph (default  2.5)
              2D coordinated are scaled by 1.+.5*Scale*(rmin-r)/rmin
              r is the distance of the vertex from the barycenter
              rmin is the smallest distance of the vertex from the barycenter 
               belonging to the outer circumferencing 5- or 6-ring
      If ScalePPG.ne.0. Scaling factor for exponential in Pisanski-Plestenjak-Graovac
              algorithm     (default 1.0)
      If PS.ne.0. 
              - Projection angle in degrees is chosen (default 45 degrees)
              to be used as an input parameter in the cone projection
              method. The angle is the cone angle from the point of projection 
              to the projection plane, which touches the point with the smallest 
              z-value (opposite the projection point). Note that angle is reset 
              to 45 deg if chosen larger than 89.
              - In the case of the perspective projection PS is the distance
              between the focal point and the ring center underneath, Note this
              is chosen by the program, but if nonzero this parameter has to be
              carefully chosen. The picture produced gives a good idea if the
              ParamS is correctly chosen or not. Note that the ring centers get
              an extra boost of 10\% in the scaling factors such that they appear
              more in the center of the rings produced by the Schlegel projection.
              
\section{Fullerene Isomer Database}
A database is provided for general isomers up to \C{100} and for IPR isomers up to
\C{120} including the number of Hamiltonian cycles. The database can be copied into
the main program folder and can be used to read the ring spiral pentagon indices.
The numbering scheme is identical to that the one chosen in the book by Fowler 
and Manolopoulus,\cite{Atlas}, that is each isomer in the book's appendix can be 
constructed easily from the database. An example is given in the input file   
{\it pentagon13.inp}. The datafiles are formatted and can easily be read. It is our 
intension to extend the isomer list beyond \C{100}/\C{120} (without Hamiltonian cycles). 
New lists will be available on our website. Note the determination of the number of
distinct Hamiltonian cycles is NP-complete and beyond 100 (120 for IPR) 
computationally too demanding. The longest file for our database ran for 3
months on a single processor.

Note: the directory database needs to be in the same directory as ``source" or ``libgraph".


%References

\begin{thebibliography}{35}

\bibitem{CYLview} C. Y. Legault, ``Program CYLview", http://www.cylview.org. 

\bibitem{Gabriel2008} A. T. Gabriel, T. Meyer, and G. Germano, 
``Molecular graphics of convex body fluids", {\it J. Chem. Theory Comput.}
{\bf 4}, 468--476 (2008). The program is available at http://qmga.sourceforge.net/.

\bibitem{Cioslowski2000} J. Cioslowski, N. Rao, D. Moncrieff, "Standard Enthalpies of Formation of Fullerenes and Their
Dependence on Structural Motifs", {\it J. Am. Chem. Soc.} {\bf 122}, 8265�8270 (2000).

\bibitem{Babic1995} D. Babi\'c, O. Ori, ``Matching polynomial and topological resonance energy of \C{70}", {\it Chem. Phys. Lett.} {\bf 234}, 240-244 (1995).

\bibitem{Babic1997} D. Babi\'c, ``Topological Resonance Energy of Fullerenes", {\it J. Chem. Inf. Comput. Sci.} {\bf 37}, 920-923 (1997).

\bibitem{Atlas} P. W. Fowler and D. E. Manolopoulus, ``An Atlas of Fullerenes" (Dover Publ., New York, 2006).
 
\bibitem{Avogadro} see http://avogadro.openmolecules.net

\bibitem{JMol} see http://jmol.sourceforge.net/

\bibitem{Pymol} see


\end{thebibliography}

\end{document}

